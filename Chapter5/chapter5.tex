%!TEX root = ../thesis.tex
%*******************************************************************************
%***************************** Fifth Chapter **********************************
%*******************************************************************************
\graphicspath{{Chapter5/Figs/Vector/}{Chapter5/Figs/}}

%%%%%%%%%%%%%%%%%%%%%%%%%%%%%%%%%%%%%%%%%%%%%%%%%%%%%%%%%%%%%%%%%%%%%%%%%%%%%%%%
% Proposed Portal Solution
%%%%%%%%%%%%%%%%%%%%%%%%%%%%%%%%%%%%%%%%%%%%%%%%%%%%%%%%%%%%%%%%%%%%%%%%%%%%%%%%
%
\chapter{Portal}

%%%%%%%%%%%%%%%%%%%%%%%%%%%%%%%%%%%%%%%%%%%%%%%%%%%%%%%%%%%%%%%%%%%%%%%%%%%%%%%%
% Introduction
%%%%%%%%%%%%%%%%%%%%%%%%%%%%%%%%%%%%%%%%%%%%%%%%%%%%%%%%%%%%%%%%%%%%%%%%%%%%%%%%
%
\section{Introduction}
% This chapter covers the actual implementation plan of connecting the pricing system with the portal frontend. How the system should behave under different sets of criteria, and how the user should be able to interact with the system. The YTA-portal should integrate the frontend that allows taxi company users to modify their pricing rules without having prior knowledge about the system.
% Ultimately, the main goal of this project is to create a system that works as the user wants it to. Freedom and broad ranges of possibilities have a cost however. Complexity confuses the user, discouraging exploration. A hand guide would ease the cognitive strain surely? Or a developer is often kind enough to take the responsibility and do the tough job for the user instead. A software system as complex as it may be, must be reasonable in the eyes of the user.
intro

%%%%%%%%%%%%%%%%%%%%%%%%%%%%%%%%%%%%%%%%%%%%%%%%%%%%%%%%%%%%%%%%%%%%%%%%%%%%%%%%
% Visual Hierarchy
%%%%%%%%%%%%%%%%%%%%%%%%%%%%%%%%%%%%%%%%%%%%%%%%%%%%%%%%%%%%%%%%%%%%%%%%%%%%%%%%
%
% - How can complex pricing rules be communicated through the UI?
% - Which views are essential?
% - In what way can visual hierarchy guide the user through processes naturally?
% - How should complex elements impacting price calculations be communicated to
%   the user through UI components?
%
\section{Visual Hierarchy}
A famous phrase often used in data visualizations called Shneiderman's mantra \cite{mantra}, lays the foundation of principles that enable a user to maintain an understanding of the context in which data is visualized. The sentences in the mantra dictate that there are three stages in data exploration.

\begin{enumerate}
	\item Overview First
	\item Zoom and Filter
	\item Details on Demand
\end{enumerate}

Pricing rules cannot be plotted in a graph, yet this mantra could be put to good use in reducing the cognitive overhead while reasoning about pricing rules. A pricing rule is nothing more than a set of prices and collection of criteria which a trip must meet in or for those prices to apply. Even if this information could be graphed, it would be highly dimensional. In order to communicate just one pricing rule, the system must guide the user from an overview down to the less important details of the rule. This is to be achieved by splitting up the various criteria into multiple views, if they require a substantial cognitive focus.

\subsection{Essential Views}
The main components that make up a the price calculation system are:

\begin{figure}[H]
	\centering
	\includegraphics[width=1\textwidth]{Treemap}
	\caption[Treemap of Components]{Treemap of Components.}
	\label{fig:Treemap}
\end{figure}

\begin{enumerate}
	\item products
	\item
\end{enumerate}
% \subsection{Entities}
% - products
% - rules
% - discounts
% - locations
% - apps
% - timeframes
% - location picker
% \subsection{Hierarchy}
% - overview pages
% - detail pages
% - composite pages

\subsection{Expressing Order}
Preattentive attributes

These attributes are what immediately catch our eye when we look at a visualization. They can be perceived in less than 10 millisec-
onds, even before we make a conscious effort to notice them.


What are we showing, when are we showing it

The central thing of the price calculation vs hierarchy

other
- internationalization

challenges
- drawing polygons
- entering hours in a week schedule
- fitting rule information, prices per products per rule and even thresholds on one page
- sorting rules and discounts

\section{Locations}
\section{Timeframes}