%!TEX root = ../thesis.tex
%*******************************************************************************
%*********************************** First Chapter *****************************
%*******************************************************************************
\graphicspath{{Chapter5/Figs/Vector/}{Chapter5/Figs/}}

%%%%%%%%%%%%%%%%%%%%%%%%%%%%%%%%%%%%%%%%%%%%%%%%%%%%%%%%%%%%%%%%%%%%%%%%%%%%%%%%
% Realization
%%%%%%%%%%%%%%%%%%%%%%%%%%%%%%%%%%%%%%%%%%%%%%%%%%%%%%%%%%%%%%%%%%%%%%%%%%%%%%%%
%
\chapter{Realization}
\section{Introduction}
During the second phase, issues from the backlog were implemented in an iterative SCRUM process. In this chapter, the final realization of the project is evaluated. Findings and observations by considering the assumptions and limitations are discussed. During development, two main applications were written. The price calculation system, and the portal that enables users to manage pricing rules in the price calculation system.

%%%%%%%%%%%%%%%%%%%%%%%%%%%%%%%%%%%%%%%%%%%%%%%%%%%%%%%%%%%%%%%%%%%%%%%%%%%%%%%%
% Methods and Techniques
%%%%%%%%%%%%%%%%%%%%%%%%%%%%%%%%%%%%%%%%%%%%%%%%%%%%%%%%%%%%%%%%%%%%%%%%%%%%%%%%
%
\section{Methods and Techniques}
In the first sprint, a project was set up in NodeJS using Typescript. The existing projects were built using Javascript, but Typescript is a much safer language, preventing bugs because the compiler can catch errors early on, warning programmers instead before the application crashes. Types also document code, expressing the intention of the programmer.

%%%%%%%%%%%%%%%%%%%%%%%%%%%%%%%%%%%%%%%%%%%%%%%%%%%%%%%%%%%%%%%%%%%%%%%%%%%%%%%%
% Sprint 1 - Dynamic Price Calculations
%%%%%%%%%%%%%%%%%%%%%%%%%%%%%%%%%%%%%%%%%%%%%%%%%%%%%%%%%%%%%%%%%%%%%%%%%%%%%%%%
%
\section{Sprint 1 - Dynamic Price Calculations}
A basic dynamic price calculation system was implemented in the first sprint, aiming to deliver a first version of the system, including fake data generators, validation of models, a single service to determine the distance and duration of a ride, rules that contain pricing information, a calculation that produces the total price of a ride using a companies rules, a formatter that produces an expected response, and tests for all of the functionalities.

%%%%%%%%%%%%%%%%%%%%%%%%%%%%%%%%%%%%%%%%%%%%%%%%%%%%%%%%%%%%%%%%%%%%%%%%%%%%%%%%
% Sprint 2 - Authentication and Authorization
%%%%%%%%%%%%%%%%%%%%%%%%%%%%%%%%%%%%%%%%%%%%%%%%%%%%%%%%%%%%%%%%%%%%%%%%%%%%%%%%
%
\section{Sprint 2 - Authentication and Authorization}
Company pricing rules can be used by applications so that each application uses a subset of the pricing rules. For this reason, TPS requires two identifiers to make a price calculation using rules for a particular company application: a companyId and a daAppInstallId. JSON Web Tokens that are signed by the core system hold the identifiers in the payload, so that the TPS can use the identifiers after decrypting the token. Companies have one country by default, which determines the currency and VAT percentage. In the breakdown, the VAT percentage is calculated from the actual price, as VAT is included. Discounts are part of the breakdown, being a percentage of the route price, or a fixed price. On top of that, it is possible that a company application uses rules that are related to a debtor, instead of its own subset of rules. Finally, the project is deployed to a staging environment so that the system could be used by the applications in the staging environment.

%%%%%%%%%%%%%%%%%%%%%%%%%%%%%%%%%%%%%%%%%%%%%%%%%%%%%%%%%%%%%%%%%%%%%%%%%%%%%%%%
% Sprint 3 - Setting up the Portal
%%%%%%%%%%%%%%%%%%%%%%%%%%%%%%%%%%%%%%%%%%%%%%%%%%%%%%%%%%%%%%%%%%%%%%%%%%%%%%%%
%
\section{Sprint 3 - Setting up the Portal}
At this point the system is fully operational, but company and daAppInstall information has to be inserted in the database manually. An endpoint is made that inserts a full company setup into the database so that prices can be calculated with five products by default. No wireframes were made beforehand, making it a task for the current sprint being executed while setting up the portal project. Angular in conjunction with Covalents UI platform is used to make the user interface, consisting of an overview and detail page for products and pricing rules. The pricing rules overview shows pricing information for each product that a company has. This is automatically maintained whenever a product is added or deleted. Furthermore, threshold rules can be added or deleted for distances and durations, making this particular view very complex. This final task was not finished in time.

%%%%%%%%%%%%%%%%%%%%%%%%%%%%%%%%%%%%%%%%%%%%%%%%%%%%%%%%%%%%%%%%%%%%%%%%%%%%%%%%
% Sprint 4 -
%%%%%%%%%%%%%%%%%%%%%%%%%%%%%%%%%%%%%%%%%%%%%%%%%%%%%%%%%%%%%%%%%%%%%%%%%%%%%%%%
%
\section{Sprint 4 - Expanding the Portal}
More screens

- processing feedback
- pricing rules overview, rules must be draggable to prioritize
- priority must be maintained distinctly
- app installations must be displayed
- allow on meter calculations
-

%%%%%%%%%%%%%%%%%%%%%%%%%%%%%%%%%%%%%%%%%%%%%%%%%%%%%%%%%%%%%%%%%%%%%%%%%%%%%%%%
% Result
%%%%%%%%%%%%%%%%%%%%%%%%%%%%%%%%%%%%%%%%%%%%%%%%%%%%%%%%%%%%%%%%%%%%%%%%%%%%%%%%
%
\section{Result}