%!TEX root = ../thesis.tex
%*******************************************************************************
%****************************** Third Chapter **********************************
%*******************************************************************************
\chapter{Proposed Approach}

% **************************** Define Graphics Path ****************************
\ifpdf
    \graphicspath{{Chapter3/Figs/Raster/}{Chapter3/Figs/PDF/}{Chapter3/Figs/}}
\else
    \graphicspath{{Chapter3/Figs/Vector/}{Chapter3/Figs/}}
\fi

% *********************************** Backend **********************************
\section{Introduction}
% Waar gaat dit hoofdstuk over?



% To answer in what way locations can be represented to be universally interpretable, the definition of a location must be well understood. Encoding of locations has historically been of great importance, and is always being modernized. This chapter aims to find the best method of representing locations that is universally interpretable and most importantly usable for the project. GPS and Postal Codes are globally adopted and have proven to be reliable methods of encoding locations. Locations, plurally, suggesting that more than one location should be encoded. This is where the challenge lies.

\section{Architectural Patterns}
\section{Database}

\section{Authentication and Authorization}
\section{Identity Management}

Which architectural pattern is best suited for implementing the pricing system?
\begin{enumerate}
    \item Which architectural patterns fit in with the exising architecture?
    \item How will authentication be handled?
    \item Which database is suited for this project?
\end{enumerate}


% *********************************** Frontend *********************************
\section{}

How can the task of defining rules be as insightful as possible to the user?
\begin{enumerate}
    \item Which views should exist, does a logical hierarchy exist among views?
    \item How should locations be defined and managed by the user?
    \item How should timeframes be handled in the interface?
\end{enumerate}

\section{Methods and Techniques}