%!TEX root = ../thesis.tex
%*******************************************************************************
%****************************** Third Chapter **********************************
%*******************************************************************************
\chapter{Proposed Portal Solution}

% **************************** Define Graphics Path ****************************
\ifpdf
    \graphicspath{{Chapter4/Figs/Raster/}{Chapter4/Figs/PDF/}{Chapter4/Figs/}}
\else
    \graphicspath{{Chapter4/Figs/Vector/}{Chapter4/Figs/}}
\fi

\section{Introduction}
% Waar gaat dit hoofdstuk over?

This chapter covers the actual implementation plan of connecting the pricing system with the portal frontend. How the system should behave under different circumstances, how the user is able to interact with the system. The YTA-portal should integrate the frontend that allows taxi company users to modify their pricing rules. This chapter aims to answer question number three, which aims to 

\section{Required Views}


\mynote{This is not researched yet, as it's covered in later sprints}

How can the task of defining rules be as insightful as possible to the user?
\begin{enumerate}
    \item Which views should exist, does a logical hierarchy exist among views?
    \item How should locations be defined and managed by the user?
    \item How should timeframes be handled in the interface?
\end{enumerate}

\section{Methods and Techniques}
