%%%%%%%%%%%%%%%%%%%%%%%%%%%%%%%%%%%%%%%%%%%%%%%%%%%%%%%%%%%%%%%%%%%%%%%%%%%%%%%%
% Tips
%%%%%%%%%%%%%%%%%%%%%%%%%%%%%%%%%%%%%%%%%%%%%%%%%%%%%%%%%%%%%%%%%%%%%%%%%%%%%%%%
%
\section{Introduction}
% Waar gaat dit hoofdstuk over?
Automatic fare estimations and calculations are one of many common features in taxi dispatch systems. A passenger books a ride, and a predicted price is displayed based on pickup and drop off locations. When the destination is reached, the system calculates the final price, with or without discounts, including taxes, and additional costs added by the driver. Taxi companies compete for customers, and prices are shifting regularly. This increases the demand for dispatch systems with easy and solid price managment features.

%%%%%%%%%%%%%%%%%%%%%%%%%%%%%%%%%%%%%%%%%%%%%%%%%%%%%%%%%%%%%%%%%%%%%%%%%%%%%%%%
% Context
%%%%%%%%%%%%%%%%%%%%%%%%%%%%%%%%%%%%%%%%%%%%%%%%%%%%%%%%%%%%%%%%%%%%%%%%%%%%%%%%
% - General
% - Structure
% - Clients
%
\subsection{Context}
This thesis is written during an assignment at taxiID, an Amsterdam based company providing end-to-end cloud solutions for taxi companies. Founded as a startup that successfully introduced smartphone taxi booking in The Netherlands, and offers a wide range of IT solutions to serve the taxi market, including a passenger app, a driver app, administrative panels, and track and trace hardware. taxiID solutions have proven to be a reliable set of tools for all size businesses. For independent taxi companies with 2 cars or a companies with large fleets, affordable solutions are available. taxiID's goal is to deliver affordable, time-saving solutions for taxi companies to allow for convenient planning and dispatching without requiring local installation. Tough based in Amsterdam, the development team is located in Medemblik, consisting of two mobile developers, two backend developers, a designer and two project managers. Clients are located across the globe, introducing challenges when developing applications that rely on clearly defined locations and infrastructures that vastly differ between countries.

%%%%%%%%%%%%%%%%%%%%%%%%%%%%%%%%%%%%%%%%%%%%%%%%%%%%%%%%%%%%%%%%%%%%%%%%%%%%%%%%
% Assignment
%%%%%%%%%%%%%%%%%%%%%%%%%%%%%%%%%%%%%%%%%%%%%%%%%%%%%%%%%%%%%%%%%%%%%%%%%%%%%%%%
% - Origin of the assignment
% - Description
% - Research question
% - Sub questions
%
% \subsection{Assignment}
% Waar gaat dit hoofdstuk over?
\subsection{Assignment}
% Waarom deze opdracht? Welke problemen moeten worden opgelost
YourDriverApp (YDA) requires a pricing calculation functionality that is similar to the existing taxiID implementation. All functionalities within the current system align with the clients demands, but some features introduce difficulties, for example: region names are too vague for specific database queries. Some features could be abstracted so more possibilities can be implemented, some features are still unimplemented, and some features could be improved along the way. A system must be implemented in which group admins can define pricing rules based on user defined locations and time schedules, that can be used for calculating a passengers trip price, or show prices of different products based on the trip the passenger is about to make. For example: a passenger may book a taxi ride from Utrecht to Schiphol using the passenger app. Available products are presented with their respective prices based on the distance and duration of the trip using the pricing rules that were created by the group admin of a taxi company. The system must be usable in countries with a poor postal code system. There should be a way for a group admin to describe locations in a way that are precise and consistent with reality, meaning that a defined location should be usable from outside of the system, or at least be interpretable. An example of this requirement would be: a taxi company that operates in Afghanistan. A passenger wants to be picked up on some road near the mountains. How would a group admin describe that location in order to define a price beforehand? The system should be accessible to other systems, meaning that applications that currently rely on the old system should be able to migrate to the new system. As the old system shouldn't be used for new applications, as it was not designed for this use case, the new system should. It should have a single responsibility, and should be atonomous in that regard.

\subsection{Research Objective}
% Hoe wordt dit onderzoek aangepakt?
% Designing research involves two separate sets of activities. The first involves determining everything you wish to achieve through the research project. This has to do with modelling the content of the research; we call this the con-ceptual design of a research project. The second set of activities concerns howto realise all this during the implementation stage of the project. This is called the technical research design.
Three main challenges that construct the assignment can be identified. Research must be done to attain the best possible way of mapping locations to pricing rules. What this means is that locations must be storable, comparable, and interpretable. The database must be able to store locations in an efficient manner, to which queries can efficiently be made in order to find out whether a pricing rule applies to a given ride. For this to be the case, the stored locations must be comparable to the location of the passenger, or the destination. The user must be able to reason about his pricing rules, from which an understanding of his defined locations logically follows.

Secondly, a system has to be developed that encapsulates the solution that is the result of the conducted research. It is helpful to extend the research of the problem to finding out how to incorporate the answers into a working system, where architecture has a major influence in the tools that are available. For example: if a solution to the main problem requires a database system capable of handling high quantities of geospatial queries, this requirement has to be satisfied in order to proceed in finding the final solution.

Finally, a portal has to be created that enables users to define the pricing rules. The complexity of the portal depends on how straight forward the price calculation system is put together. The best way of making the systems capabilities available to the user through the UI in the portal, must be investigated.

\subsection{Questions}
% Welke hoofdvraag moet worden beantwoord om de opdracht succesvol af te ronden?

From the description of the problem, one main important research question can be derived: \\

\textit{How can a generic location-based price calculation system be implemented that is usable around the globe?} \\

% Hoe kunnen we de hoofdvraag opdelen?
Which varieties of address formats exist around the world?

Is it possible to redefine different address formats as a generic address?

\begin{itemize}
  \item Gps
  \item Postal code
  \item Intersecting areas
  \item The hotel case
  \item The airport case
\end{itemize}

To what extent do address formats have an impact on performance of a price calculation?

