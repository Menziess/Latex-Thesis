%%%%%%%%%%%%%%%%%%%%%%%%%%%%%%%%%%%%%%%%%%%%%%%%%%%%%%%%%%%%%%%%%%%%%%%%%%%%%%%%
% Preface
%%%%%%%%%%%%%%%%%%%%%%%%%%%%%%%%%%%%%%%%%%%%%%%%%%%%%%%%%%%%%%%%%%%%%%%%%%%%%%%%
%
\section*{Preface}
% \addcontentsline{toc}{section}{Preface}
Before you lies the graduation report that displays all the accomplishments and research conducted during the final phase of my Bachelor Software Engineering study at the Amsterdam University of Applied Sciences, written to fulfill the graduation requirements. Allow me to briefly elaborate on the events that motivated me to reach this point in my career.

Before I began my study, I was a marine in the Royal Dutch Marine Corps. As their latin motto qua patet orbis ("As Far as the World Extends") suggests, I was sent to exotic places around the world. During these trips there was either lots of waiting, or lots of hard and dangerous work. A common activity during the waiting hours of the average marine was watching series, but I wasn’t really into that. Before I was sent to Afghanistan, I bought a new laptop on which I installed visual studio and downloaded C++ tutorials, convinced that I would be learning how to program during my off-duty hours. My efforts, however sincere, didn't convince me that I was progressing my understanding of the matter very much. During one of my boat trips along the coast of West-Africa I tried once more, but this time I started experimenting with Javascript instead of C++, yielding more tangible results.

After five years I decided to quit the marines because I felt that I could accomplish greater things with a proper education, and went to an open house of the HvA, HBO-ICT. There I was confronted with my interest for logic and programming once more, and decided that I was going to become a Software Engineer. During my time at the HvA, I enjoyed the logic and elegance of algorithms and patterns in code very much, next to solving complex problems, and learning new things every day in general. I’ve learned some shallow concepts about machine learning in the Big Data course which interested me the most. Systems that could learn complex tasks for themselves.

Right before I started writing this thesis, I followed a minor Artificial Intelligence at the University of Amsterdam, and right before that I worked at taxiID as a vacation job. That’s where I was offered an internship where I would develop a virtual assistant. This aligned very well with my minor and my interest in artificial intelligence in general. However, because of priorities within the company, the assignment changed to rebuilding a price calculation system from scratch that had to be used around the globe, qua patet orbis.

To many this may sound as a boring challenge. If you are not one of those people, you may think about the many ways in which this problem may be solved. And this thesis will provide useful solutions for encoding locations and handling geospatial data in an intelligent and performant way. I was also challenged by the fact that my project had to supercede the existing system in effectiveness and efficiency. The moment I was introduced to my assignment reminded me of a chapter in a book called Clean Code, in which this exact pursuit is used as a common example for which the book would provide solutions; “Now the two teams are in a race. The tiger team must build a new system that does everything that the old system does. Not only that, they have to keep up with the changes that are continuously being made to the old system. Management will not replace the old system until the new system can do everything that the old system does." \cite{b0}.

I would like to thank Dan Stefancu, Marco Strijker and Martin Zwaneveld for their insightful criticism that has led to the most useful lessons during my internship.

Stefan Schenk

Andijk, 01-03-2018