%!TEX root = ../thesis.tex
%*******************************************************************************
%*********************************** First Chapter *****************************
%*******************************************************************************
\graphicspath{{Chapter1/Figs/Vector/}{Chapter1/Figs/}}

%%%%%%%%%%%%%%%%%%%%%%%%%%%%%%%%%%%%%%%%%%%%%%%%%%%%%%%%%%%%%%%%%%%%%%%%%%%%%%%%
% Introduction
%%%%%%%%%%%%%%%%%%%%%%%%%%%%%%%%%%%%%%%%%%%%%%%%%%%%%%%%%%%%%%%%%%%%%%%%%%%%%%%%
% - Introduction to the introduction
% - Context
% - Restatement of the problem
% - Restatement of the response
% - Roadmap
%
\chapter{Introduction}
What was once an ordinary startup known as Uber, is now the most famous taxi dispatch company in the world \cite{Uber}. A similar startup was founded in the same year, called taxiID; an Amsterdam based company providing end-to-end cloud solutions and mobile applications for taxi companies. Hailing a taxi has rarely been performed by sticking out ones hand, hoping to catch the attention of a bypassing taxi driver ever since. The ability to order a cab lies at everyone's fingertips, literally. Recently, taxiID has started developing a new brand called YourDriverApp (YDA), a lighter and newer version of the original solution, being more focussed on smaller taxi companies. Despite the fact that YDA is new, it still depends on the price calculation functionality of the legacy system. This chapter expands on how this issue is translated into the assignment.

%%%%%%%%%%%%%%%%%%%%%%%%%%%%%%%%%%%%%%%%%%%%%%%%%%%%%%%%%%%%%%%%%%%%%%%%%%%%%%%%
% Context
%%%%%%%%%%%%%%%%%%%%%%%%%%%%%%%%%%%%%%%%%%%%%%%%%%%%%%%%%%%%%%%%%%%%%%%%%%%%%%%%
% - General
% - Clients
% - Structure
% - Cooperation
%
\section{Context}
taxiID was founded as a startup that successfully introduced smartphone taxi booking in The Netherlands, offering a wide variety of IT solutions to serve the taxi market, including a passenger app, a driver app, and administrative panels. More specifically: an app for passengers to order a taxi, an app for drivers to receive their job assignments, and services for all size businesses, offering convenient planning and dispatching without requiring local installations. Businesses that make use of taxiID's services can be found anywhere in the world. This introduces complicated challenges when developing applications that depend on countries' infrastructure and postal code systems. The development team responsible for solving these problems is located in Medemblik. Consisting of two mobile app developers (iOS and Android), two backend developers, two designers and a project managers.

%%%%%%%%%%%%%%%%%%%%%%%%%%%%%%%%%%%%%%%%%%%%%%%%%%%%%%%%%%%%%%%%%%%%%%%%%%%%%%%%
% Problem Definition
%%%%%%%%%%%%%%%%%%%%%%%%%%%%%%%%%%%%%%%%%%%%%%%%%%%%%%%%%%%%%%%%%%%%%%%%%%%%%%%%
% - How does it work now
% - Why is that a problem
%
\section{Problem Definition}
The new YDA apps depend on the price calculation module that is part of the legacy system, for which it was once designed and implemented. Taxi companies have prices defined for the vehicle rides that they offer to passengers. If YDA clients want to price their products, they are obligated to use the legacy taxiID portal, which has to store company information in a platform that is not related to YDA. This makes little sense, as much as it is efficient from a technical point of view, as well as being easy to maintain and extend. The legacy price calculation module knows three types of pricing rules: fixed prices based on postal codes or addresses, tier prices based on kilometer thresholds, and dynamic calculations based on the distance and duration of a ride. These rules will be selected depending on the characteristics of a taxi trip, but they will be matched in the same order as mentioned. A company may have as many pricing rules as required, only one rule will be used to calculate the final price. The fixed rules are defined by downloading, modifying, and uploading a .csv file as presented in Table \ref{tab:fixedprices}, the other types of rules are simply managed through a web form.

\begin{table}[htbp!]
	\centering
	\begin{tabular}{c|c|c|c|c}
		\toprule
		Departure & Destination & Nr Passengers & Price & Vehicle Type \\
		\midrule
		1462      & 1313        & 4             & 125   & ...          \\
		1313      & 1462        & 4             & 125   & ...          \\
		1462      & 1313        & 8             & 150   & ...          \\
		1313      & 1462        & 8             & 150   & ...          \\
		1462      & 1012        & 4             & 65    & ...          \\
		1012      & 1462        & 4             & 65    & ...          \\
		0         & 1462        & 4             & 65    & ...          \\
		1462      & 0           & 4             & 65    & ...          \\
		1462      & AIR1        & 4             & 89    & ...          \\
		AIR1      & 1462        & 4             & 89    & ...          \\
		\bottomrule
	\end{tabular}
	\caption[Legacy CSV Pricing Definition File]{Comma Separated File containing Fixed Prices in cents.}
	\label{tab:fixedprices}
\end{table}

Matching postal codes has been no problem in the Netherlands, but internationalizing would pose problems with wide range of postal code and address formats. Some countries do not have a postal code, which would prevent some companies from being able to use YDA. When a passenger books a ride, the price calculation module will first compare the postal codes and/or addresses, amount of passengers, and desired vehicle types that are sent by the booking app with the fixed pricing rules. A fixed price is returned as soon as a match is found. If no match is found in any of the fixed pricing rules, the system proceeds to calculate a price using a kilometer threshold rule, given that at least one exists. This type of calculation decreases or increases the price per kilometer for every successive amount of kilometers that have surpassed a predetermined threshold. This concept will be discussed in chapter 4. If this rule does not exist, a dynamic rule is used to calculate the price based on distance and duration of the ride. Finally, on top of the prices that have been calculated, a discount may be applied as a fixed amount, as a percentage of the price, or as a so called alternative fixed pricing table. When this last option is selected, the price will be calculated all over again, using a newly referenced fixed pricing rule. This process is not just hard to understand for a user, who has to reason about the companies' prices. But it is also hard to understand for programmers, who have to maintain the code that supports this functionality. A small mistake in the csv file could lead to major issues if the error passes through deployment undetected.

%%%%%%%%%%%%%%%%%%%%%%%%%%%%%%%%%%%%%%%%%%%%%%%%%%%%%%%%%%%%%%%%%%%%%%%%%%%%%%%%
% Assignment
%%%%%%%%%%%%%%%%%%%%%%%%%%%%%%%%%%%%%%%%%%%%%%%%%%%%%%%%%%%%%%%%%%%%%%%%%%%%%%%%
%
\section{Assignment}
The title of this thesis reads:

\[\textit{"A rule-based geospatial reasoning system for trip price calculations"}.\] \hfill

A Trip Pricing System (TPS) must be designed and implemented to calculate trip prices based on user defined pricing rules. Concisely, YourDriverApp requires its own pricing calculation functionality that is similar to the existing taxiID implementation but must not be incorporated into a non-related monolithic, highly coupled system, as it is today. Clients must be able to set up pricing rules and discounts through the YDA portal. It is important that the feature that allows clients to define locations for the pricing rules, are usable in countries that have no workable postal code system.

%%%%%%%%%%%%%%%%%%%%%%%%%%%%%%%%%%%%%%%%%%%%%%%%%%%%%%%%%%%%%%%%%%%%%%%%%%%%%%%%
% Research
%%%%%%%%%%%%%%%%%%%%%%%%%%%%%%%%%%%%%%%%%%%%%%%%%%%%%%%%%%%%%%%%%%%%%%%%%%%%%%%%
% - How to get the required knowledge
%
\section{Research}
Four main challenges that construct the assignment can be identified. Research must be conducted to attain the best possible way of mapping locations to pricing rules, while keeping the locations conveyable. An alternative technique to comparing and storing locations must be explored, that does not regress in comparison to the old technique, that uses CSV files and postal codes. Edge cases must be dealt with to prevent a situation in which users are unable to achieve that which they could with the old system. The best way of integrating the new system in the existing system architecture should be investigated. But improvements must not be withheld, proposals with good arguments should be presented with regards to the chosen technologies, authentication, authorization, and system design. The price calculation algorithm must be examined, including the logic of matching rules. The best way to communicate this logic to the user through the YDA portal in a way that makes reasoning about the price definitions possible, must be found. All the functionalities mentioned must be usable anywhere in the world.

\subsection{Questions}
From the description of the problem, one main important research question can be derived:

\[\textit{How can a generic location-based price calculation system be implemented}\]
\[\textit{that is usable in every country?}\] \hfill

This question encapsulates the four important challenges that have to be dealt with before the project can successfully be implemented. In order to give a clear direction to the research, sub-questions are separated into four groups; location mapping, architecture, trip pricing system, and user interface.

\begin{enumerate}
	\item Which location encoding is sufficient for this system to be operational?
	\begin{enumerate}[label*=\arabic*.]
		\item How can legacy location definitions be improved to be universally interpretable?
		\item In what way can location matching be improved?
		\item Which Database Management Systems are candidate for handling this project's use cases?
	\end{enumerate}

	\item What is most fitting solution to integrate the backend and frontend into the existing architecture?
	\begin{enumerate}[label*=\arabic*.]
		\item Which architectural patterns fit in with the existing system architecture?
		\item How is state shared and synchronized between system components?
		\item What is the most applicable authentication method?
	\end{enumerate}

	\item Which logic and data is required in the backend to reliably calculate a trip price?
	\begin{enumerate}[label*=\arabic*.]
		\item Which criteria should regulate whether rules match?
		\item How can determinism of price computations be guaranteed?
		\item In what way can the three original pricing rule types be implemented? (fixed, dynamic, and threshold prices)
	\end{enumerate}

	\item Is it possible to communicate the inner workings of the system through the user interface?
	\begin{enumerate}[label*=\arabic*.]
		\item Which backend concepts are essential to display in the frontend?
		\item Which design practices allow users to understand coherence of different elements that make up a rule?
		\item How should a user know what the outcome of his interactions with the system are?
	\end{enumerate}

\end{enumerate}

Answering these questions will lead to the implementation of a solid, straightforward, user-friendly system that utilizes the user interface to communicate the inner-workings of the rule-based price calculation system.

%%%%%%%%%%%%%%%%%%%%%%%%%%%%%%%%%%%%%%%%%%%%%%%%%%%%%%%%%%%%%%%%%%%%%%%%%%%%%%%%
% Process
%%%%%%%%%%%%%%%%%%%%%%%%%%%%%%%%%%%%%%%%%%%%%%%%%%%%%%%%%%%%%%%%%%%%%%%%%%%%%%%%
% - How to complete the final product
% - Steps taken before implementation phase
%
\section{Process}
A desire from within taxiID to use the SCRUM methodology to potentially improve their development process is an important factor to set up this project in a way that would introduce the team to SCRUM without forcing developers and CEO's to adopt it right away. All team members are familiarized with tools, roles, workflows, and the project artifacts somewhat indirectly. Because of the novelty of SCRUM in regard to the product owner, a pregame phase is introduced for preparation purposes, see Table \ref{tab:planning}. A written working method is provided to the product owner, see Appendix \ref{appendix:pregame}, Phase I - Pregame. The interpretation of the product owners product vision and the reflection from a developer viewpoint is documented, so that miscommunications and misinterpretations can be resolved before the project is started. It contains an architectural vision and a proposed solution, which is agreed upon by the product owner before the backlog is created. Reading the document is recommended if more knowledge about the process and context of the assignment is desired.

\begin{table}[htbp!]
	\centering
	\begin{tabular}{ccc|c|c|c|c|c|c|c|c}
		\toprule
		\multicolumn{3}{c}{Phase I - Pregame}   &
		\multicolumn{8}{c}{Phase II - Game}  		\\
		\midrule
		\rotate{week 1}                         &
		\rotate{week 2}                         &
		\rotate{week 3}                         &
		\rotate{week 4}                         &
		\rotate{week 5}                         &
		\rotate{week 6}                         &
		\rotate{week 7}                         &
		\rotate{week 8}                         &
		\rotate{week 9}                         &
		\rotate{week 10}                        &
		\rotate{week 11}                        \\
		\midrule
		\rotate{product definition}     				&
		\rotate{architectural vision} 					&
		\rotate{proposed solution}							&
		\rotate{sprint 1}                       &
		\rotate{sprint 2}                       &
		\rotate{sprint 3}                       &
		\rotate{sprint 4}                       &
		\rotate{sprint 5}                       &
		\rotate{sprint 6}                       &
		\rotate{sprint 7}                       &
		\rotate{sprint 8}                       \\
		\bottomrule
	\end{tabular}
	\caption[Project Roadmap]{Project roadmap.}
	\label{tab:planning}
\end{table}

\begin{figure}[htbp!]
	\centering
	\includegraphics[width=.9\textwidth]{Organogram}
	\caption[Organogram]{Organizational chart of taxiID.}
	\label{fig:latlngsphere}
\end{figure}